% TeX root = ../Main.tex

% First argument to \section is the title that will go in the table of contents. Second argument is the title that will be printed on the page.
\section[Lecture 1 (Date) -- {\it Basic Concepts}]{Lecture 1 (1/21)}

\subsection{Constraints}
A constraint can be defined as a rule, parameter or a limitation that the system on which it is imposed
on must follow. Constraints can be classified into the following: 
    \subsubsection{Holonomic}
    If the equation of constraint acting on a system of N particles can be expressed in the form $f(r_{1}, r_{2}, r_{3},....)=0$
    relating the coordinates and time, the constraint is called a Holonomic constraint. 
    \subsubsection{Non-Holonomic}
    If the constraint cannot be expressed in the $f(r_{1}, r_{2}, r_{3},....)=0$ form and involves
    an inequality sign, the constraint is called Non-Holonomic.
    \subsubsection{Scleronomic}
    The Constraints that are independent of time are classified as scleronomic constraints.
    \subsubsection{Rheonomic}
    Constraints that contain time as an explicit variable, i.e dependant on time are called rheonomic constraints.
\subsection{Generalized coordinates}
For a system of N particles, 3N coordinates are required to define it completely. If there are 'k' equations of Constraints, the 
number of degreese of freedom are reduced to $3N-k=n$. Thus we require N independent coordinates $\{q_{1}, q_{2}, .... , q_{n}\}$
to specify define the system. This set of N coordinates are called generalized coordinates. The transformation equation of the 
system can be expressed in terms of Generalized equations as 
$$r_{i}=r_{j}\{q_{1}, q_{2}, q_{3}, ... , t\}$$ 
Where j=1, 2, 3 \dots

For a particle moving on the surface of a sphere, generalized coordinates are $\{\theta, \phi\}$ as the system has two degrees of freedom.
For a bird flying through the sky, the generalized coordinates is $\{x, y, z\}$. 

\subsection{Second subsection}