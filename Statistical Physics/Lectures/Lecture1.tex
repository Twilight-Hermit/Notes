% TeX root = ../Main.tex

% First argument to \section is the title that will go in the table of contents. Second argument is the title that will be printed on the page.
\section[Lecture 1 (Date) -- {\it Molecular Energies in an Ideal Gas}]{Lecture 1 (1/21)}

\subsection{Introduction}
Maxwell-Boltzmann statistics is used to find the energy distribution of moleclues in an Ideal Gas.
Since quantization of energy due to translational motion is very less and the number of molecules N is 
very large, it is reeasonable to consider a continous distribution instead of discrete where energy of each 
molecule is added up.

\subsection{Main:}
To get the distribution, we must first find the number of molecules that have the energy in the range E and E+dE(I'm using E instead of the symbol prof used as I don't know what it is, tell me if you do.)
$$n(E)d(E)$$
And to find this, we must know that number of states that have energy between E and E+due
$$g(E)d(E)$$
We'll use momentum to do so, 
$$p=\sqrt{2mE}=\sqrt{(p_{x})^2+(p_{y})^2+(p_{z})^2}$$
Consider a momentum space with axes $p_{x}$, $p_{y}$, $p_{z}$

The number of states with the momentum between p and p+dp is equivalent to volume of a sphericla shell of radius p and thickness dp.
Since the formula for volume is $4{\pi}p^2dp$(derivative of volume of sphere), 
$$g(p)dp=Bp^2dp$$
Where B is a constant, and since each momentum magnitude corresponds to a single energy state, 
$$g(E)dE=Bp^2dp$$
since $p^2=2mE$ and $dp=\frac{mdE}{\sqrt{2mE}}$
$$g(E)d{E}=2m^{3/2}B\sqrt{E}dE$$

